% Options for packages loaded elsewhere
\PassOptionsToPackage{unicode}{hyperref}
\PassOptionsToPackage{hyphens}{url}
%
\documentclass[
]{article}
\usepackage{amsmath,amssymb}
\usepackage{iftex}
\ifPDFTeX
  \usepackage[T1]{fontenc}
  \usepackage[utf8]{inputenc}
  \usepackage{textcomp} % provide euro and other symbols
\else % if luatex or xetex
  \usepackage{unicode-math} % this also loads fontspec
  \defaultfontfeatures{Scale=MatchLowercase}
  \defaultfontfeatures[\rmfamily]{Ligatures=TeX,Scale=1}
\fi
\usepackage{lmodern}
\ifPDFTeX\else
  % xetex/luatex font selection
\fi
% Use upquote if available, for straight quotes in verbatim environments
\IfFileExists{upquote.sty}{\usepackage{upquote}}{}
\IfFileExists{microtype.sty}{% use microtype if available
  \usepackage[]{microtype}
  \UseMicrotypeSet[protrusion]{basicmath} % disable protrusion for tt fonts
}{}
\makeatletter
\@ifundefined{KOMAClassName}{% if non-KOMA class
  \IfFileExists{parskip.sty}{%
    \usepackage{parskip}
  }{% else
    \setlength{\parindent}{0pt}
    \setlength{\parskip}{6pt plus 2pt minus 1pt}}
}{% if KOMA class
  \KOMAoptions{parskip=half}}
\makeatother
\usepackage{xcolor}
\usepackage[margin=1in]{geometry}
\usepackage{color}
\usepackage{fancyvrb}
\newcommand{\VerbBar}{|}
\newcommand{\VERB}{\Verb[commandchars=\\\{\}]}
\DefineVerbatimEnvironment{Highlighting}{Verbatim}{commandchars=\\\{\}}
% Add ',fontsize=\small' for more characters per line
\usepackage{framed}
\definecolor{shadecolor}{RGB}{248,248,248}
\newenvironment{Shaded}{\begin{snugshade}}{\end{snugshade}}
\newcommand{\AlertTok}[1]{\textcolor[rgb]{0.94,0.16,0.16}{#1}}
\newcommand{\AnnotationTok}[1]{\textcolor[rgb]{0.56,0.35,0.01}{\textbf{\textit{#1}}}}
\newcommand{\AttributeTok}[1]{\textcolor[rgb]{0.13,0.29,0.53}{#1}}
\newcommand{\BaseNTok}[1]{\textcolor[rgb]{0.00,0.00,0.81}{#1}}
\newcommand{\BuiltInTok}[1]{#1}
\newcommand{\CharTok}[1]{\textcolor[rgb]{0.31,0.60,0.02}{#1}}
\newcommand{\CommentTok}[1]{\textcolor[rgb]{0.56,0.35,0.01}{\textit{#1}}}
\newcommand{\CommentVarTok}[1]{\textcolor[rgb]{0.56,0.35,0.01}{\textbf{\textit{#1}}}}
\newcommand{\ConstantTok}[1]{\textcolor[rgb]{0.56,0.35,0.01}{#1}}
\newcommand{\ControlFlowTok}[1]{\textcolor[rgb]{0.13,0.29,0.53}{\textbf{#1}}}
\newcommand{\DataTypeTok}[1]{\textcolor[rgb]{0.13,0.29,0.53}{#1}}
\newcommand{\DecValTok}[1]{\textcolor[rgb]{0.00,0.00,0.81}{#1}}
\newcommand{\DocumentationTok}[1]{\textcolor[rgb]{0.56,0.35,0.01}{\textbf{\textit{#1}}}}
\newcommand{\ErrorTok}[1]{\textcolor[rgb]{0.64,0.00,0.00}{\textbf{#1}}}
\newcommand{\ExtensionTok}[1]{#1}
\newcommand{\FloatTok}[1]{\textcolor[rgb]{0.00,0.00,0.81}{#1}}
\newcommand{\FunctionTok}[1]{\textcolor[rgb]{0.13,0.29,0.53}{\textbf{#1}}}
\newcommand{\ImportTok}[1]{#1}
\newcommand{\InformationTok}[1]{\textcolor[rgb]{0.56,0.35,0.01}{\textbf{\textit{#1}}}}
\newcommand{\KeywordTok}[1]{\textcolor[rgb]{0.13,0.29,0.53}{\textbf{#1}}}
\newcommand{\NormalTok}[1]{#1}
\newcommand{\OperatorTok}[1]{\textcolor[rgb]{0.81,0.36,0.00}{\textbf{#1}}}
\newcommand{\OtherTok}[1]{\textcolor[rgb]{0.56,0.35,0.01}{#1}}
\newcommand{\PreprocessorTok}[1]{\textcolor[rgb]{0.56,0.35,0.01}{\textit{#1}}}
\newcommand{\RegionMarkerTok}[1]{#1}
\newcommand{\SpecialCharTok}[1]{\textcolor[rgb]{0.81,0.36,0.00}{\textbf{#1}}}
\newcommand{\SpecialStringTok}[1]{\textcolor[rgb]{0.31,0.60,0.02}{#1}}
\newcommand{\StringTok}[1]{\textcolor[rgb]{0.31,0.60,0.02}{#1}}
\newcommand{\VariableTok}[1]{\textcolor[rgb]{0.00,0.00,0.00}{#1}}
\newcommand{\VerbatimStringTok}[1]{\textcolor[rgb]{0.31,0.60,0.02}{#1}}
\newcommand{\WarningTok}[1]{\textcolor[rgb]{0.56,0.35,0.01}{\textbf{\textit{#1}}}}
\usepackage{graphicx}
\makeatletter
\def\maxwidth{\ifdim\Gin@nat@width>\linewidth\linewidth\else\Gin@nat@width\fi}
\def\maxheight{\ifdim\Gin@nat@height>\textheight\textheight\else\Gin@nat@height\fi}
\makeatother
% Scale images if necessary, so that they will not overflow the page
% margins by default, and it is still possible to overwrite the defaults
% using explicit options in \includegraphics[width, height, ...]{}
\setkeys{Gin}{width=\maxwidth,height=\maxheight,keepaspectratio}
% Set default figure placement to htbp
\makeatletter
\def\fps@figure{htbp}
\makeatother
\setlength{\emergencystretch}{3em} % prevent overfull lines
\providecommand{\tightlist}{%
  \setlength{\itemsep}{0pt}\setlength{\parskip}{0pt}}
\setcounter{secnumdepth}{-\maxdimen} % remove section numbering
\ifLuaTeX
  \usepackage{selnolig}  % disable illegal ligatures
\fi
\usepackage{bookmark}
\IfFileExists{xurl.sty}{\usepackage{xurl}}{} % add URL line breaks if available
\urlstyle{same}
\hypersetup{
  pdftitle={Group Project 1},
  hidelinks,
  pdfcreator={LaTeX via pandoc}}

\title{Group Project 1}
\usepackage{etoolbox}
\makeatletter
\providecommand{\subtitle}[1]{% add subtitle to \maketitle
  \apptocmd{\@title}{\par {\large #1 \par}}{}{}
}
\makeatother
\subtitle{Biology 368/664 Bucknell University}
\author{}
\date{\vspace{-2.5em}14 Sep 2024}

\begin{document}
\maketitle

This project will require you to develop a tutorial to teach Bucknell
students how to use R for graphing and data analysis.

\subsection{Introduction}\label{introduction}

Begin by introducing yourself to your group. Then discuss the biggest
challenge that you have faced during the first three weeks of this
course. Determine if there are any common threads in these challenges
and start to think about objectives for the tutorial.

\subsection{Target Audience}\label{target-audience}

Discuss with your group the target audience for the tutorial. Examples
could be a new student in Biology 364/664, a student in one of the new
core Biology classes (201, 202, 203, or 204), a student in another
300-level Biology course (not 364), or a new student in one of the
Bucknell research groups.

Edit the README.md file in your group's repository to reflect your plan.

\subsection{Objectives}\label{objectives}

After deciding on the target audience for your tutorial determine 2 to 3
overall objectives for your tutorial (one per member of your group).
These should be high-level objectives that are important skills for your
target audience. Check with Prof.~Field to see if they are appropriate
and then add them to your README.md file.

Identify at least 2 goals within each objective and add them to your
README.md file. These should be goals that someone who is working
through the tutorial can self-asses. For example, ``to demostrate that
you can test a hypothesis using a statistical model, the student should
use a T test, linear model, or other test and interpret the p value
appropriately.''

\subsection{Grading}\label{grading}

Each group will be expected to complete the following tasks to earn 85\%
of the points available for this assignment (21/25).

\begin{itemize}
\tightlist
\item
  Identify and obtain suitable dataset
\item
  Use a Github repository and version control to collaborate on the
  project

  \begin{itemize}
  \tightlist
  \item
    Every member of the group should participate in editing the repo
  \end{itemize}
\item
  Spend 4-6 hours preparing, coding, and testing tutorial

  \begin{itemize}
  \tightlist
  \item
    Data exploration
  \item
    Data visualization
  \item
    Hypothesis testing
  \end{itemize}
\item
  Present tutorial in class
\item
  Provide public archive suitable for sharing to students/faculty
\end{itemize}

Tutorials from previous classes can be viewed at our public github site:
\url{https://github.com/Bucknell-Biol364}

Each group should use an \emph{Acknowledgements} section to document the
participation of each member and the collaboration within and between
groups.

Additional credit will be awarded for providing assistance to other
groups or for the development of a tutorial that goes beyond the minimal
expectations listed above. You will have the opportunity to provide
feedback to another group after the initial deadline (like for Homework
02).

\subsection{Sample Dataset}\label{sample-dataset}

One of the possible datasets to use for the tutorial can be found in the
datasauRus package.

\begin{Shaded}
\begin{Highlighting}[]
\NormalTok{datasaurus\_dozen }\SpecialCharTok{|\textgreater{}} 
  \FunctionTok{group\_by}\NormalTok{(dataset) }\SpecialCharTok{|\textgreater{}} 
  \FunctionTok{summarize}\NormalTok{(}
      \AttributeTok{mean\_x    =} \FunctionTok{mean}\NormalTok{(x),}
      \AttributeTok{mean\_y    =} \FunctionTok{mean}\NormalTok{(y),}
      \AttributeTok{std\_dev\_x =} \FunctionTok{sd}\NormalTok{(x),}
      \AttributeTok{std\_dev\_y =} \FunctionTok{sd}\NormalTok{(y),}
      \AttributeTok{corr\_x\_y  =} \FunctionTok{cor}\NormalTok{(x, y)}
\NormalTok{      )}
\end{Highlighting}
\end{Shaded}

\begin{verbatim}
## # A tibble: 13 x 6
##    dataset    mean_x mean_y std_dev_x std_dev_y corr_x_y
##    <chr>       <dbl>  <dbl>     <dbl>     <dbl>    <dbl>
##  1 away         54.3   47.8      16.8      26.9  -0.0641
##  2 bullseye     54.3   47.8      16.8      26.9  -0.0686
##  3 circle       54.3   47.8      16.8      26.9  -0.0683
##  4 dino         54.3   47.8      16.8      26.9  -0.0645
##  5 dots         54.3   47.8      16.8      26.9  -0.0603
##  6 h_lines      54.3   47.8      16.8      26.9  -0.0617
##  7 high_lines   54.3   47.8      16.8      26.9  -0.0685
##  8 slant_down   54.3   47.8      16.8      26.9  -0.0690
##  9 slant_up     54.3   47.8      16.8      26.9  -0.0686
## 10 star         54.3   47.8      16.8      26.9  -0.0630
## 11 v_lines      54.3   47.8      16.8      26.9  -0.0694
## 12 wide_lines   54.3   47.8      16.8      26.9  -0.0666
## 13 x_shape      54.3   47.8      16.8      26.9  -0.0656
\end{verbatim}

Compare the means and standard deviations of the 13 different datasets.

Boxplots of either the x or the y value show that there are some
differences, even though the means and standard deviations are
identical.

\begin{Shaded}
\begin{Highlighting}[]
\NormalTok{datasaurus\_dozen }\SpecialCharTok{|\textgreater{}}
  \FunctionTok{ggplot}\NormalTok{(}\FunctionTok{aes}\NormalTok{(}\AttributeTok{x =}\NormalTok{ x, }\AttributeTok{colour =}\NormalTok{ dataset)) }\SpecialCharTok{+}
    \FunctionTok{geom\_boxplot}\NormalTok{() }\SpecialCharTok{+}
    \FunctionTok{theme\_void}\NormalTok{() }\SpecialCharTok{+}
    \FunctionTok{theme}\NormalTok{(}\AttributeTok{legend.position =} \StringTok{"none"}\NormalTok{) }\SpecialCharTok{+}
    \FunctionTok{facet\_wrap}\NormalTok{(}\SpecialCharTok{\textasciitilde{}}\NormalTok{dataset, }\AttributeTok{ncol =} \DecValTok{3}\NormalTok{)}
\end{Highlighting}
\end{Shaded}

\includegraphics{Project01_files/figure-latex/unnamed-chunk-2-1.pdf}

\begin{Shaded}
\begin{Highlighting}[]
\NormalTok{datasaurus\_dozen }\SpecialCharTok{|\textgreater{}}
  \FunctionTok{ggplot}\NormalTok{(}\FunctionTok{aes}\NormalTok{(}\AttributeTok{x =}\NormalTok{ y, }\AttributeTok{colour =}\NormalTok{ dataset))}\SpecialCharTok{+}
    \FunctionTok{geom\_boxplot}\NormalTok{()}\SpecialCharTok{+}
    \FunctionTok{theme\_void}\NormalTok{()}\SpecialCharTok{+}
    \FunctionTok{theme}\NormalTok{(}\AttributeTok{legend.position =} \StringTok{"none"}\NormalTok{)}\SpecialCharTok{+}
    \FunctionTok{facet\_wrap}\NormalTok{(}\SpecialCharTok{\textasciitilde{}}\NormalTok{dataset, }\AttributeTok{ncol =} \DecValTok{3}\NormalTok{)}
\end{Highlighting}
\end{Shaded}

\includegraphics{Project01_files/figure-latex/unnamed-chunk-3-1.pdf}

But you have to visualize all of the data with a scatter plot to really
see the patterns.

\begin{Shaded}
\begin{Highlighting}[]
\NormalTok{datasaurus\_dozen }\SpecialCharTok{|\textgreater{}} 
  \FunctionTok{ggplot}\NormalTok{(}\FunctionTok{aes}\NormalTok{(}\AttributeTok{x =}\NormalTok{ x, }\AttributeTok{y =}\NormalTok{ y, }\AttributeTok{colour =}\NormalTok{ dataset))}\SpecialCharTok{+}
    \FunctionTok{geom\_point}\NormalTok{()}\SpecialCharTok{+}
    \FunctionTok{theme\_void}\NormalTok{()}\SpecialCharTok{+}
    \FunctionTok{theme}\NormalTok{(}\AttributeTok{legend.position =} \StringTok{"none"}\NormalTok{)}\SpecialCharTok{+}
    \FunctionTok{facet\_wrap}\NormalTok{(}\SpecialCharTok{\textasciitilde{}}\NormalTok{dataset, }\AttributeTok{ncol =} \DecValTok{3}\NormalTok{)}
\end{Highlighting}
\end{Shaded}

\includegraphics{Project01_files/figure-latex/unnamed-chunk-4-1.pdf}

And did you notice the code in the \{r\} codechunk header that
controlled the size of the output in the Rmd? Pretty neat trick!

And here are two versions of the data that you could use in your data
visualization tutorial. To use them you would probably want to change
the names of the datasets and also make x and y more meaningful. Then
save them as a csv or tsv to be imported later for your tutorial.

\begin{Shaded}
\begin{Highlighting}[]
\NormalTok{datasaurus\_long }\OtherTok{\textless{}{-}}\NormalTok{ datasaurus\_dozen}
\NormalTok{datasaurus\_wide }\OtherTok{\textless{}{-}}\NormalTok{ datasaurus\_dozen\_wide}
\FunctionTok{head}\NormalTok{(datasaurus\_long)}
\end{Highlighting}
\end{Shaded}

\begin{verbatim}
## # A tibble: 6 x 3
##   dataset     x     y
##   <chr>   <dbl> <dbl>
## 1 dino     55.4  97.2
## 2 dino     51.5  96.0
## 3 dino     46.2  94.5
## 4 dino     42.8  91.4
## 5 dino     40.8  88.3
## 6 dino     38.7  84.9
\end{verbatim}

\begin{Shaded}
\begin{Highlighting}[]
\FunctionTok{head}\NormalTok{(datasaurus\_wide)}
\end{Highlighting}
\end{Shaded}

\begin{verbatim}
## # A tibble: 6 x 26
##   away_x away_y bullseye_x bullseye_y circle_x circle_y dino_x dino_y dots_x
##    <dbl>  <dbl>      <dbl>      <dbl>    <dbl>    <dbl>  <dbl>  <dbl>  <dbl>
## 1   32.3   61.4       51.2       83.3     56.0     79.3   55.4   97.2   51.1
## 2   53.4   26.2       59.0       85.5     50.0     79.0   51.5   96.0   50.5
## 3   63.9   30.8       51.9       85.8     51.3     82.4   46.2   94.5   50.2
## 4   70.3   82.5       48.2       85.0     51.2     79.2   42.8   91.4   50.1
## 5   34.1   45.7       41.7       84.0     44.4     78.2   40.8   88.3   50.6
## 6   67.7   37.1       37.9       82.6     45.0     77.9   38.7   84.9   50.3
## # i 17 more variables: dots_y <dbl>, h_lines_x <dbl>, h_lines_y <dbl>,
## #   high_lines_x <dbl>, high_lines_y <dbl>, slant_down_x <dbl>,
## #   slant_down_y <dbl>, slant_up_x <dbl>, slant_up_y <dbl>, star_x <dbl>,
## #   star_y <dbl>, v_lines_x <dbl>, v_lines_y <dbl>, wide_lines_x <dbl>,
## #   wide_lines_y <dbl>, x_shape_x <dbl>, x_shape_y <dbl>
\end{verbatim}

\section{Acknowledgements}\label{acknowledgements}

DatasauRus package and description below: Stephanie Locke
\url{https://github.com/jumpingrivers/datasauRus}

The datasauRus package wraps the awesome Datasaurus Dozen dataset, which
contains 13 sets of x-y data. Each sub-dataset has five statistics that
are (almost) the same in each case. (These are the mean of x, mean of y,
standard deviation of x, standard deviation of y, and Pearson
correlation between x and y). However, scatter plots reveal that each
sub-dataset looks very different. The dataset is intended to be used to
teach students that it is important to plot their own datasets, rather
than relying only on statistics.

The Datasaurus was created by Alberto Cairo and described in the paper
\href{https://www.autodeskresearch.com/publications/samestats}{Same
Stats, Different Graphs: Generating Datasets with Varied Appearance and
Identical Statistics through Simulated Annealing} by Justin Matejka and
George Fitzmaurice.

In the paper, Justin and George simulate a variety of datasets that the
same summary statistics to the Datasaurus but have very different
distributions.

This package also makes these datasets available for use as an advanced
\href{https://en.wikipedia.org/wiki/Anscombe\%27s_quartet}{Anscombe's
Quartet}, available in R as \texttt{anscombe}.

\end{document}
